%%%%%%%%%%%%%%%%%%%%%%%%%%%%%%%%%%%%%%%%%
% Jacobs Landscape Poster
% LaTeX Template
% Version 1.1 (14/06/14)
%
% Created by:
% Computational Physics and Biophysics Group, Jacobs University
% https://teamwork.jacobs-university.de:8443/confluence/display/CoPandBiG/LaTeX+Poster
%
% Further modified by:
% Nathaniel Johnston (nathaniel@njohnston.ca)
%
% This template has been downloaded from:
% http://www.LaTeXTemplates.com
%
% License:
% CC BY-NC-SA 3.0 (http://creativecommons.org/licenses/by-nc-sa/3.0/)
%
%%%%%%%%%%%%%%%%%%%%%%%%%%%%%%%%%%%%%%%%%

%----------------------------------------------------------------------------------------
%	PACKAGES AND OTHER DOCUMENT CONFIGURATIONS
%----------------------------------------------------------------------------------------

%% Path to affiliate logos
%%
%% NOTE: THIS MUST BE INCLUDED BEFORE THE MENTION OF confposter
\newcommand\logodir{C:/Users/jln54/Documents/ARS/Writing/Resources/Logos/}
% \newcommand\logodir{C:/Users/jeffrey.neyhart/OneDrive - USDA/Documents/CranberryLab/Jeff/Presentations/PresentationResources/}

\newcommand\logopathA{\logodir/USDA_Logos/USDAARSIdentityPantone3.png}
\newcommand\logopathB{\logodir/Rutgers_Logos/SEBSDPB_S_RED_BLACK.png}
\newcommand\logopathC{\logodir/Neyhart_lab_logo_text.png}



%% Define the paths to the BibTex file for the Zotero library
\newcommand\bibdir{C:/Users/jln54/Documents/ARS/Writing/Resources/}
\newcommand\biblibrary{\bibdir/Paperpile_bibtex_20231101}


\documentclass[final]{beamer}

% Use the beamerposter package for laying out the poster
\usepackage[scale=1.24,orientation=portrait]{beamerposter}
% \usepackage[scale=1.24,orientation=landscape]{beamerposter}

\usetheme{confposter} % Use the confposter theme supplied with this template

\usepackage{wrapfig} % Use for wrapping figures

\usepackage[super,numbers]{natbib} % Use for fine-tunning the captions


% Packages to support table creation
\usepackage{booktabs}
\usepackage{longtable}
\usepackage{array}
\usepackage{multirow}
\usepackage{wrapfig}
\usepackage{float}
\usepackage{colortbl}
\usepackage{pdflscape}
\usepackage{tabu}
\usepackage{threeparttable}
\usepackage{threeparttablex}
\usepackage[normalem]{ulem}
\usepackage[utf8]{inputenc}
\usepackage{makecell}
\usepackage{xcolor}


%% Configure code blocks
\usepackage{listings} % Use for code blocks
\usepackage{color}

\definecolor{dkgreen}{rgb}{0,0.6,0}
\definecolor{gray}{rgb}{0.5,0.5,0.5}
\definecolor{mauve}{rgb}{0.58,0,0.82}

\lstset{frame=tb,
  language=Java,
  aboveskip=3mm,
  belowskip=3mm,
  showstringspaces=false,
  columns=flexible,
  basicstyle={\small\ttfamily},
  numbers=none,
  numberstyle=\tiny\color{gray},
  keywordstyle=\color{blue},
  commentstyle=\color{dkgreen},
  stringstyle=\color{mauve},
  breaklines=true,
  breakatwhitespace=true,
  tabsize=3
}

% \usepackage{biblatex} % Bibliography

\setbeamercolor{block title}{fg=greenset1,bg=white} % Colors of the block titles
\setbeamercolor{block body}{fg=black,bg=white} % Colors of the body of blocks
\setbeamercolor{block alerted title}{fg=white,bg=blueset1!70} % Colors of the highlighted block titles
\setbeamercolor{block alerted body}{fg=black,bg=blueset1!10} % Colors of the body of highlighted blocks

% Many more colors are available for use in beamerthemeconfposter.sty

%-----------------------------------------------------------

% Define the column widths and overall poster size
% To set effective sepwid, onecolwid and twocolwid values, first choose how many columns you want and how much separation you want between columns
% In this template, the separation width chosen is 0.02 of the paper width and a 3-column layout
% onecolwid should therefore be (1-(ncol + 1)*sepwid)/ncol e.g. (1-(3+1)*0.024)/3 = 0.3013333
% Set twocolwid to be (2*onecolwid)+sepwid = 0.6266666
% Set threecolwid to be (3*onecolwid)+2*sepwid = 0.9279999

\newlength{\sepwid}
\newlength{\onecolwid}
\newlength{\twocolwid}
\newlength{\threecolwid}

%% Poster size
\setlength{\paperwidth}{46in} % A0 width: 46.8in
\setlength{\paperheight}{33in} % A0 height: 33.1in


\setlength{\sepwid}{0.024\paperwidth} % Separation width (white space) between columns
\setlength{\onecolwid}{0.3013333\paperwidth} % Width of one column
\setlength{\twocolwid}{0.6266666\paperwidth} % Width of two columns
\setlength{\threecolwid}{0.9279999\paperwidth} % Width of three columns
\setlength{\topmargin}{-1in} % Reduce the top margin size
%-----------------------------------------------------------

\usepackage{graphicx}  % Required for including images

\usepackage{booktabs} % Top and bottom rules for tables

\usepackage{array}

%----------------------------------------------------------------------------------------
%	TITLE SECTION
%----------------------------------------------------------------------------------------

\title{Stage-wise analysis of multi-year phenotypic data for cranberry genotypic evaluation} % Poster title
% Add \\ to induce new line

\vspace{1cm}

% Author(s)
\author{ \textbf{Jeffrey L. Neyhart\textsuperscript{1,2}}, Jennifer Johnson-Cicalese\textsuperscript{2}, and Juan E. Zalapa\textsuperscript{3,4} }

% Affiliations
\institute{
  \textsuperscript{1}USDA-ARS, Chatsworth, NJ.;
  \textsuperscript{2}P.E. Marucci Center for Blueberry and Cranberry Research and Extension, Chatsworth, NJ.;
  \textsuperscript{3}USDA-ARS, Madison, WI.;
  \textsuperscript{4}Dept. of Horticulture, University of Wisconsin-Madison, Madison, WI.
}

% *Contact: Email - \href{mailto:jeffrey.neyhart@usda.gov}{jeffrey.neyhart@usda.gov}


%----------------------------------------------------------------------------------------

\begin{document}

\addtobeamertemplate{block end}{}{\vspace*{2ex}} % White space under blocks
\addtobeamertemplate{block alerted end}{}{\vspace*{2ex}} % White space under highlighted (alert) blocks

\setlength{\belowcaptionskip}{2ex} % White space under figures
\setlength\belowdisplayshortskip{2ex} % White space under equations

\begin{frame}[t] % The whole poster is enclosed in one beamer frame

\begin{columns}[t] % The whole poster consists of three major columns, the second of which is split into two columns twice
% the [t] option aligns each column's content to the top

%----------------------------------------------------------------------------------------

\begin{column}{\sepwid}\end{column} % Empty spacer column

\begin{column}{\onecolwid} % The first column


%----------------------------------------------------------------------------------------
%	TAKEAWAYS
%----------------------------------------------------------------------------------------

\setbeamercolor{block alerted title}{fg=white,bg=orangeset1!70} % Change the alert block title colors
\setbeamercolor{block alerted body}{fg=black,bg=orangeset1!10} % Change the alert block body colors

\begin{alertblock}{\Large{Takeaways}}

% \begin{large}
\begin{itemize}
  \item \textbf{Analyzing multi-environment trial data for perennial crops (e.g. cranberry) must account for serial correlations caused by repeated measures}
  \vspace{0.5cm}
  \item \textbf{Genetic correlations between years were high for four cranberry traits}
  \vspace{0.5cm}
  \item \textbf{Accounting for serial correlations led to better model fit and higher prediction accuracy}
\end{itemize}
% \end{large}


\end{alertblock}

%----------------------------------------------------------------------------------------
%	INTRODUCTION
%----------------------------------------------------------------------------------------

\begin{block}{1. Introduction}
\vspace{-2cm}


Optimal analysis of multi-environment trials (METs) can aid in identifying superior genotypes in breeding programs\cite{Bernardo2010-iw}.

\vspace{0.5cm}

Stage-wise analysis is common for analyzing data from METs of annual crops\cite{Smith2019-fg}.

\vspace{0.5cm}

MET data for perennial crops like cranberry (\textit{Vaccinium macrocarpon} Ait.) presents statistical challenges:
\begin{itemize}
  \item \textbf{Unbalanced data}: not all genotype-environment combinations are observed
  \item \textbf{Serial correlation from repeated measures}: plots, blocks, and genotypes are not randomized (and are thus correlated) across years\cite{Piepho2013-kv}
\end{itemize}

\vspace{0.5cm}

% % Figure 1
%
\begin{center}
  \begin{figure}
    \includegraphics[width=0.8\linewidth]{figures/figure1.png}
    % \caption{A figure!}
  \end{figure}
\end{center}


% % Figure 2
%
\begin{center}
  \begin{figure}
    \includegraphics[width=0.8\linewidth]{figures/figure2_compiled.png}
    \caption{\textit{Fruit of the zoom (in)}: a bed, plot, and fruit from cranberry breeding trials}
  \end{figure}
\end{center}





\vspace{0.5cm}

\textbf{Objectives:}
\begin{itemize}
  \item Identify the best two-stage models for analyzing data from cranberry METs.
  \item Compare the ability of models to predict genotypic performance for multiple traits in future years.
\end{itemize}


\end{block}



% %----------------------------------------------------------------------------------------
% %	OBJECTIVES OR QUESTIONS
% %----------------------------------------------------------------------------------------
%
% % Set the color
% \setbeamercolor{block alerted title}{fg=white,bg=blueset1!70} % Change the alert block title colors
% \setbeamercolor{block alerted body}{fg=black,bg=blueset1!10} % Change the alert block body colors
%
% \begin{alertblock}{\large{Objective}}
%
% \begin{itemize}
%   \item{\textbf{Identify genomic regions associated with local environmental factors in wild cranberry populations.}}
%   % \item{\textbf{Determine candidate abiotic stress tolerance loci as potential selection targets.}}
%   % \item \textbf{What genomic regions are associated with local environmental conditions in wild cranberry?}
%   % \item What is the frequency and occurrance of putatively adaptive loci in native cranberry selections?
% \end{itemize}
%
%
% \end{alertblock}


% \vspace{1cm}


% %----------------------------------------------------------------------------------------
% %	MATERIALS AND MATERIALS
% %----------------------------------------------------------------------------------------
%
% \begin{block}{Phenotypic data}
%
%
% % End the M&M block
% \end{block}
%
%

%----------------------------------------------------------------------------------------

\end{column} % End of the first column (i.e. Introduction, objectives, and methods)

%----------------------------------------------------------------------------------------

% Begin the results columns

\begin{column}{\sepwid}\end{column} % Empty spacer column

\begin{column}{\twocolwid} % Begin a column which is two columns wide (column 2)


% Removing the two-columns within the middle column structure
% \begin{columns}[t,totalwidth=\twocolwid] % Split up the two columns wide column
%
% \begin{column}{\onecolwid}\vspace{-.6in} % The first column within column 2 (column 2.1)

%----------------------------------------------------------------------------------------
%	RESULTS
%----------------------------------------------------------------------------------------

\begin{block}{2. Phenotypic Data and Stage 1 Analysis}


%% Columns for figure 3



%% Split into two columns
\begin{columns}[t,totalwidth=\twocolwid] % Split up the two columns wide column


% First column is for phenotypic data
\begin{column}{0.33\twocolwid}
\vspace{-2cm}



We analyzed phenotypic data for four traits in two multi-year cranberry evaluation trials:




%% Table 1: description of phenotyping trials

\begin{scriptsize}

\begin{table}
\centering
\begin{tabular}{>{\raggedright\arraybackslash}p{1in}>{\centering\arraybackslash}p{1.5in}>{\centering\arraybackslash}p{1.2in}>{\centering\arraybackslash}p{1.7in}>{\centering\arraybackslash}p{1in}>{\centering\arraybackslash}p{1.8in}}
\toprule
\textbf{Trial} & \textbf{Year\ Planted} & \textbf{Plots} & \textbf{Genotypes} & \textbf{Reps} & \textbf{Years\ Harvested}\\
\midrule
AYT03 & 2003 & 240 & 80 & 3 & 5\\
AYT13 & 2013 & 60 & 30 & 2 & 6\\
\bottomrule
\end{tabular}
\end{table}

\end{scriptsize}




%% Table 1: description of traits

\begin{scriptsize}

\begin{table}
\centering
\begin{tabular}{>{\raggedright\arraybackslash}p{2in}>{\centering\arraybackslash}p{1in}>{\centering\arraybackslash}p{5.5in}}
\toprule
\textbf{Trait} & \textbf{Abbrv.} & \textbf{Description}\\
\midrule
Fruit Weight & FW & Average mass (g) of individual fruit \\
Fruit Yield & FY & Total yield of fruit (Mg) per unit area (ha) \\
Total Anthocyanin Content & TAC & Level of anthocyanins (mg) per mass (g) of fruit \\
Percent Fruit Rot & PFR & Percentage (\%) of rotted fruit in a harvested sample \\
\bottomrule
\end{tabular}
\end{table}

\end{scriptsize}


\end{column}



% Second column is a figure
\begin{column}{0.45\twocolwid}
\vspace{-2cm}

\begin{footnotesize}

\textbf{Breeding plots are not randomized each year, which induces correlations}

\end{footnotesize}


% Figure 3
\begin{center}
  \begin{figure}
    \includegraphics[width=0.9\linewidth]{figures/figure2a.png}
    % \caption{A figure!}
  \end{figure}
\end{center}

\end{column}




% \textbf{Breeding plots are not randomized each year, which induces correlations}





% The last column describes the Stage 1 procedure and results
\begin{column}{0.22\twocolwid}
\vspace{-2cm}



\begin{footnotesize}

\begin{itemize}
  \item{In \textbf{Stage 1}, genotype-harvest year means (BLUEs) are estimated using a mixed-model\cite{Endelman2023-rk}.}
  \item{BLUEs and their variances are saved for Stage 2.}
  \item{\textbf{Modeling the correlation of spatial (plot) effects over years improves fit.}}
\end{itemize}


\end{footnotesize}


\begin{scriptsize}

% Table of model results

\begin{table}
\centering
\begin{tabular}{lccc}
\toprule
\multicolumn{1}{c}{ } & \multicolumn{1}{c}{ } & \multicolumn{2}{c}{Model AIC} \\
\cmidrule(l{3pt}r{3pt}){3-4}
Trial & Trait & Null & Correlation\\
\midrule
AYT03 & FW & 0 & -30.3\\
AYT03 & FY & 0 & -93.7\\
AYT03 & TAC & 0 & -14.6\\
AYT03 & PFR & 0 & -40.0\\
AYT13 & FW & 0 & -6.8\\
AYT13 & FY & 0 & -5.7\\
AYT13 & TAC & 0 & -8.1\\
AYT13 & PFR & 0 & -42.7\\
\bottomrule
\end{tabular}
\end{table}


\end{scriptsize}



\end{column}


\end{columns}

% End the stage 1 block
\end{block}




%----------------------------------------------------------------------------------------

% Split into two columns
\begin{columns}[t,totalwidth=\twocolwid] % Split up the two columns wide column

% First column for stage two block
\begin{column}{0.67\twocolwid}


% \vspace{2cm}


% Begin the stage 2 block
\begin{block}{3. Stage 2 Analysis}


%% Columns for the stage 2 block

%% Split into two columns
\begin{columns}[t,totalwidth=0.67\twocolwid] % Split up the two columns wide column

% First column
\begin{column}{0.335\twocolwid}
\vspace{-2cm}


% Describe the model
\begin{footnotesize}

In \textbf{Stage 2}, genotype-harvest year BLUPs are predicted using a model:

\begin{equation*}
BLUE[G_{ij}] = y_{ij} = H_j + g(H)_{ij} + s_{ij}
\end{equation*}

$y_{ij}$: BLUE of the genotype-harvest mean \\
$H_j$: fixed effect of harvest year \\
$g(H)_{ij}$: BLUP of genotype in harvest year [$g(H)_{ij} \sim N(0, \mathbf{G})$] \\
$s_{ij}$: residual error


\vspace{1cm}

\textbf{Genetic correlations were high between harvest years (upper-right triangle in the below $\mathbf{G}$ matrices):}

\end{footnotesize}

% Add a figure
% Figure 3
\begin{center}
  \begin{figure}
    \includegraphics[width=0.90\linewidth]{figures/figure3.png}
    % \caption{A figure!}
  \end{figure}
\end{center}




% \begin{scriptsize}
%
%
% \begin{table}
% \centering
% \begin{tabular}{>{\raggedright\arraybackslash}p{1.5in}>{\centering\arraybackslash}p{
% 6in}}
% \toprule
% Cov. Str. & Description\\
% \midrule
% ID & Constant genetic variance across years\\
% CS & Constant genetic variance across years and constant correlation between years\\
% CSH & Heterogenous genetic variance across years and constant correlation between years\\
% AR1 & Constant genetic variance across years and decaying correlation between years\\
% AR1H & Heterogenous genetic variance across years and decaying correlation between years\\
% \bottomrule
% \end{tabular}
% \end{table}
%
% \end{scriptsize}



\end{column}


% Second column
\begin{column}{0.335\twocolwid}
\vspace{-2cm}


\begin{footnotesize}

\textbf{Cross-validation:}

\begin{itemize}
  \item{Phenotypes in later years were masked}
  \item{Phenotypes in early years were used in prediction}
  \item{\textbf{Models that include serial correlations were often most accurate}}
\end{itemize}

% Add a figure
% Figure 3
\begin{center}
  \begin{figure}
    \includegraphics[width=0.80\linewidth]{figures/figure5.png}
    % \caption{A figure!}
  \end{figure}
\end{center}


\end{footnotesize}


%
% \begin{scriptsize}
%
%
% \begin{table}
% \centering
% \begin{tabular}{lcccccc}
% \toprule
% \multicolumn{1}{c}{ } & \multicolumn{1}{c}{ } & \multicolumn{5}{c}{RMSEP} \\
% \cmidrule(l{3pt}r{3pt}){3-7}
% Trial & Trait & ID & CS & CSH & AR1 & AR1H\\
% \midrule
% AYT03 & FW & 0.243 & \textbf{0.203} & 0.216 & 0.204 & 0.217\\
% AYT03 & FY & 112 & \textbf{102} & 101 & 119 & 118\\
% AYT03 & PFR & 16.4 & 16.6 & 16.5 & \textbf{16.1} & 16.6\\
% AYT03 & TAC & 5.13 & 12.5 & 12.5 & \textbf{4.10} & 11.3\\
% AYT13 & FW & 0.232 & 0.375 & 0.375 & \textbf{0.193} & 0.373\\
% AYT13 & FY & 98.7 & 89.4 & \textbf{89.4} & NA & 108\\
% AYT13 & PFR & 14.3 & 14.3 & \textbf{12.9} & 14.3 & 14.3\\
% AYT13 & TAC & \textbf{15.1} & 17.7 & 15.5 & 16.8 & 17.4\\
% \bottomrule
% \end{tabular}
% \end{table}
%
% \end{scriptsize}



\end{column}

% End the Stage 2 columns
\end{columns}


% End the Stage 2 block
\end{block}


% End the column containing stage 2
\end{column}


%----------------------------

% Open a second column for reference/acknolwedgements
\begin{column}{0.33\twocolwid}




%----------------------------------------------------------------------------------------
%	CONCLUSIONS AND NEXT STEPS
%----------------------------------------------------------------------------------------


% Change the font of the block title for the next two sections
% \setbeamerfont{block title}{size=\small}

\setbeamercolor{block alerted title}{fg=white,bg=orangeset1!70} % Change the alert block title colors
\setbeamercolor{block alerted body}{fg=black,bg=orangeset1!10} % Change the alert block body colors

\vspace{4cm}

\begin{alertblock}{\Large{4. Next Steps}}

\begin{footnotesize}
\begin{itemize}
  \item \textbf{Compare serial correlation models in genomic prediction}
  \vspace{0.5cm}
  \item \textbf{Evaluate sparse phenotyping designs}
  \vspace{0.5cm}
  \item \textbf{Develop improved perennial breeding trial experimental designs}
\end{itemize}
\end{footnotesize}


\end{alertblock}





%----------------------------------------------------------------------------------------
%	ACKNOWLEDGEMENTS
%----------------------------------------------------------------------------------------


% Change the font of the block title for the next two sections
% \setbeamerfont{block title}{size=\small}

\begin{block}{\large{Acknowledgements}}
\vspace{-2cm}

\begin{footnotesize}

We thank Dr. Nicholi Vorsa for sharing data from evaluation trials from the Rutgers Univ. cranberry breeding program This research was supported by USDA NIFA SCRI Grant 2022-51181-38322 and USDA-ARS project 8042-21000-023-000D. USDA is an equal opportunity provider and employer.

\end{footnotesize}

\end{block}


%----------------------------------------------------------------------------------------
%	REFERENCES
%----------------------------------------------------------------------------------------

\begin{block}{\large{References}}
\vspace{-2cm}


% \nocite{*} % Insert publications even if they are not cited in the poster
% ^ removed this so list only those publications that are cited

% \setbeamercolor{

\begin{scriptsize}

\bibliographystyle{supporting_files/posterbibstyle}
\bibliography{\biblibrary}


\end{scriptsize}

% End the reference block
\end{block}


% end the second column
\end{column}


% end the two columns containing stage 2 and ref/ack
\end{columns}


%----------------------------------------------------------------------------------------

%----------------------------------------------------------------------------------------
% End the second main column
\end{column} % End of the third column
%----------------------------------------------------------------------------------------



\end{columns} % End of all the columns in the poster

\end{frame} % End of the enclosing frame

\end{document} % End the poster document
